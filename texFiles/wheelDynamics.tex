\documentclass[letterpaper, 10pt, conference]{ieeeconf}
%\documentclass[10pt,english,a4paper]{IEEEtran}
\IEEEoverridecommandlockouts
\overrideIEEEmargins

%\usepackage[a4paper, total={14cm, 25cm}]{geometry}
\usepackage{blindtext}
\usepackage[utf8]{inputenc}
\usepackage{amsmath}
\usepackage{amssymb}
\usepackage{color}
\usepackage{graphicx}
\usepackage[pdfencoding=auto]{hyperref}
%\usepackage{natbib}
\usepackage{bm}
\usepackage{graphicx}
\usepackage{comment}
\usepackage{soul}
\usepackage{pdfpages}
%\usepackage[pdftex]{graphicx}
\usepackage{subfigure}
\pdfminorversion=4


\sethlcolor{green}

\input{abbr.tex}

\def\ankle{\text{ankle}}
\newcommand{\TODO}[1]{{\color{red} {\bf TODO:} {#1}}}
\newcommand{\SC}[1]{{\color{red} {\bf SC:} {#1}}}
\renewcommand{\SS}[1]{{\color{blue} {\bf SC:} {#1}}}
\newcommand\norm[1]{\left\lVert#1\right\rVert}
%\renewcommand\IEEEkeywordsname{Keywords}

\def\pcmd{p^{cmd}}
\def\vcmd{v^{cmd}}

\title{\Large \bf
	Wheel Kinematics and Dynamics}

\author{Saeid Samadi, and Abderrahmane Kheddar
	\thanks{Authors are with Montpellier Laboratory of Informatics,
		Robotics and Microelectronics (LIRMM), CNRS-University of Montpellier, France.
		Corresponding author: {\tt\footnotesize saeid.samadi@lirmm.fr}}
}

\begin{document}

\maketitle
\thispagestyle{empty}
\pagestyle{empty}

\begin{abstract}
	This study deals with the kinematic and dynamic model of wheels. Wheels are part of robot model and this modeling has a significant effect in the performance of the robot...
\end{abstract}

\begin{keywords}
	Wheeled Robot Model, Dynamics, Kinematics
\end{keywords}

	
%%% XXX To be uncommented on "ieeeConf" document class
%\begin{keywords}
%Chebyshev center
%\end{keywords}
\begin{comment}

\begin{table}[h]
\renewcommand{\arraystretch}{1.5}
\caption{Table of Symbols}
\label{table_example}
\centering
\begin{tabular}{p{0.09\textwidth}|p{0.09\textwidth}||p{0.09\textwidth}|p{0.09\textwidth}}
%\begin{tabular*}{0.4\textwidth}{l | r || l | r}
\hline
\bfseries \centering First & \bfseries Next & me & you \\
\hline\hline
1.0 & 2.0\\
\hline
\end{tabular}
\end{table}
\end{comment}

\section{Introduction} \label{Sec_Introduction}
Wheel kinematics and dynamics are widely studied. Considering the multi-directional slippages, due to numerous variables such as wheel inertia, material and contact forces within these models, are challenging.  In sections~\ref{Sec_Kinematics}~and~\ref{Sec_Dynamics}, we describe the proposed method in~\cite{Seegmiller2014thesis}.
\section{Wheel Kinematics} \label{Sec_Kinematics}

\section{Wheel Dynamics} \label{Sec_Dynamics}

\bibliographystyle{ieeetr}
\bibliography{refs}
\end{document}
